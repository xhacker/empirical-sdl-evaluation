\documentclass{article}

\usepackage{inconsolata}
\usepackage{algorithm}
\usepackage{amsmath}
\usepackage{graphicx,epsfig}
\usepackage[noend]{algpseudocode}
\usepackage[toc,page]{appendix}
\usepackage{float}
\usepackage[margin=2cm]{geometry}
\usepackage{url}

\newcommand{\graphWidth}{11.5cm}

\title{\textbf{An Empirical Evaluation of the Statement Deletion Mutation Operator in Real-World Python Projects}}
\author{Shan Cao, Dongyuan Liu}

% ATTENTION, READER
% The filecontents thing here is where you put sources.
% Then, use, eg, \cite{GP} to cite the @misc{GP... item.
% This does mean compiling is now 3 steps IF you've added new sources:
% $ pdflatex report.tex
% This, as a side effect, generates our bib file.
% $ bibtex report
% This compiles the bib file
% $ pdflatex report.tex
% This incorporates the new bibliography into the report
% If you haven't changed sources, just use pdflatex report.tex once, it'll be fine.

\begin{document}

\bibliographystyle{unsrt}
\bibliography{references}

\maketitle
\tableofcontents

\section{Introduction}

Many open-source libraries use test coverage as a measurement of test suite quality. However, test coverage only guarantee the code is being executed when running the tests, it does not check that the tests are able to detect real faults. An extreme example is a test suite without any assertion.

To get deeper knowledge of the flaws, running test set against slightly modified versions of programs is one approach\cite{}. This technique is called \emph{mutation testing}. A quality measure, \emph{mutation adequacy score}, is used to assess. The mutants are generated from operations such as insert and delete. However, generating and testing a vast amount of mutants can be slow, therefore the efficiency of each operation should be analyzed to approximate the goal methodically.

using only delete is effective and fast ref 1 2 3

\subsection{Related Work}

The efficiency and effectiveness of only applying delete operation has been investigated in existing works.

\subsection{Our Contribution}

we evaluate on real-world Python projects

\section{Implementation Details}

\subsection{MutPy}

\emph{MutPy} is a mutation testing tool for Python projects \cite{mutpy}.

\subsection{Statement Deletion Mutation Operator}

\section{Empirical Evaluation}

\section{Conclusions}

\subsection{Further Work}
some thing

\begin{appendices}
\end{appendices}

\bibliography{\jobname}

\end{document}
